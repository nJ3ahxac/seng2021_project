\documentclass{article}
\usepackage{enumitem}
\usepackage{graphicx}
\usepackage{grffile}
\usepackage{float}
\begin{document}
\title{%
 Deliverable 2\\
 \large An analysis and selection of software architecture, internal design\\
 \large and interface composition
}
\author{Epic Group}
\date{}
\maketitle

\section*{Introduction}
The purpose of this report is to clarify and justify the design components of 
CinemaScout, a web-based movie recommendation platform. The report will comprise
of the following components:
\subsubsection*{Part 1}
\begin{enumerate}
\item \textbf{External Data Sources}: An enumaration over the third-party data
sources CinemaScout will access.
\item \textbf{Software Components}: A list of the major software components
that will be required for CinemaScout to function.
\item \textbf{Component Choices}: A justification of the choice of software
languages for all compoents of CinemaScout.
\item \textbf{Machine Requirements}: A justification of the machine requirements
of CinemaScout.
\item \textbf{Summary}: A summary of the entailments the aforementioned
software architecture decisions provide.
\end{enumerate}
\subsubsection*{Part 2}
\begin{enumerate}
\item \textbf{User Stories}: An updated list of user stories relating to the
previous report.
\item \textbf{Sequence Diagrams}: An interaction diagram for each use case
of CinemaScout as described in the user stories.
\end{enumerate}
\section{Software Architecture}
\subsection{External Data Sources}
CinemaScout is a movie aggregator which requires comprehensive movie 
related information to function.
Foremost, we require an encompassing and comprehensive list of all
movies to date. In addition, the user expects each movie to be listed alongisde 
its creation date, title and rating among other title specific information. 
This will require an API (application programming interface) which provides 
such data for all movies in the database of CinemaScout.\newline \newline
There exists no comprehensive API which singly provides all information 
CinemaScout requires. As a result, we are required to source data from multiple
external databases:
\begin{itemize}
\item \textbf{IMDb}: The Interactive Movie Database provides archive
snapshots of its comprehensive movie data sets. However, the data provided
is limited in detail as it only contains metadata for each movie in the IMDb
database. The dataset of IMDb is sufficient in providing a list of all movies,
although specific details such as rating and plot are not provided.
\item \textbf{OMDb}: The Open Movie Database is a RESTful web service
which provides comprehensive movie information for a singular, specific title. 
OMDb completes our data requirements, compensating for the lack of detail
provided by IMDb.
\end{itemize}
Most critically, the information provided by the database snapshots of IMDb
are a requisite of making a query to OMDb. As a result, the two external data
sources complement each other and complete our data sourcing requirements.
\subsection{Software Components}
\subsection{Component Choices}
placeholder
\subsection{Machine Requirements}
placeholder
\subsection{Summary}
placeholder
\section{Initial Software Design}
\subsection{User Stories}
placeholder
\subsection{Sequence Diagrams}
placeholder
\end{document}
